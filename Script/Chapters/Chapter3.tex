% Chapter 1
% !TeX spellcheck = en_US 
\chapter{Machine Learning Models} % Main chapter title

\label{Chapter3} % For referencing the chapter elsewhere, use \ref{Chapter1} 
\setcounter{chapter}{3}
%\setcounter{section}{0}

\section{Neural Networks}





\chapter{Kernel based approximation}


Kernel based methods used to be one of the main tools for machine learning up about 15 years ago, 
when neural networks became more popular.
Nowadays they are still in use due to their rich theoretical background. 
Recent interest was also spiked due to the so-called neural tangent kernel (NTK).

The first part on the following sections on kernel based approximation in mainly based on \cite{wendland2005scattered},
while the last part on the neural tangent kernel is based on recent literature \cite{jacot2018neural}.


\section{Introduction \& Motivation}

We start by discussing the following problem:

\begin{problem}
\label{prob:interpolation_problem}
Consider $\Omega \subset \R^d$ and pairwise distinct points $X_N := \{ x_i \}_{i=1}^N \subset \Omega$ and target values $\{ f_i \}_{i=1}^N \subset \R$.
Find a continuous function $s: \Omega \rightarrow \R$ that interpolates these values, i.e.\
\begin{align}
\label{eq:interpol_conditions}
s(x_i) = f_i, \qquad i=1, ..., N.
\end{align}
\end{problem}

There are two key important ingredients:
\begin{itemize}
\item Multivariate: We are not only interested in the univariate case $d=1$, but especially in the multivariate case $d>1$.
\item Meshless: The points $X_N$ may be arbitrarily unstructured and scattered, in particular we do not need any mesh.
\end{itemize}

As a first example to solve this problem, we can consider the univariate case $d=1$:
For the approximation of our problem, we pick an ansatz space $V$ with basis $\{ \phi_j \}_{j=1}^N \subset V$ and use the ansatz function
\begin{align*}
s(x) = \sum_{j=1}^N \alpha_j \phi_j(x).
\end{align*}
This ansatz function can be plugged into the system of equations from Eq.~\eqref{eq:interpol_conditions}, 
which yields the linear equation system
\begin{align}
\label{eq:lin_eq_system_general}
A_{\phi, X_N} \alpha := 
\begin{pmatrix}
\phi_1(x_1) & \phi_2(x_1) & \dots & \phi_N(x_1) \\
\phi_1(x_2) & \phi_2(x_2) & \dots & \phi_N(x_2) \\
\vdots & \vdots & \vdots & \vdots \\
\phi_1(x_N) & \phi_2(x_N) & \dots & \phi_N(x_N) \\
\end{pmatrix}
\begin{pmatrix}
\alpha_1 \\ \alpha_2 \\ \vdots \\ \alpha_N
\end{pmatrix}
=
\begin{pmatrix}
f_1 \\ f_2 \\ \vdots \\ f_N
\end{pmatrix}.
\end{align}
In order to make this more explicit, we pick the space of polynomials of degree $N-1$ as our ansatz space, 
i.e.\ $V := \pi_{N-1}(\R)$,
and as basis we choose the monomials $\{ \phi_j \}_{j=1}^N = \{ x^{j-1} \}_{j=1}^N \subset \pi_{N-1}(\R)$.
Thus the linear equation system Eq.~\eqref{eq:lin_eq_system_general} turns into
\begin{align*}
A_{\phi, X_N} \alpha := 
\begin{pmatrix}
1 & x_1 & \dots & x_1^{N-1} \\
1 & x_2 & \dots & x_2^{N-1} \\
\vdots & \vdots & \vdots & \vdots \\
1 & x_N & \dots & x_N^{N-1} \\
\end{pmatrix}
\begin{pmatrix}
\alpha_1 \\ \alpha_2 \\ \vdots \\ \alpha_N
\end{pmatrix}
=
\begin{pmatrix}
f_1 \\ f_2 \\ \vdots \\ f_N
\end{pmatrix}.
\end{align*}
The occuring matrix $A_{\phi, X_N}$ is the so-called Vandermonde matrix, and we recall that its determinant is given as
\begin{align*}
\det(A_{\phi, X_n}) = \prod_{1 \leq i < j \leq N} (x_j - x_i).
\end{align*}
Like this we can easily see, that the linear equation system Eq.~\eqref{eq:lin_eq_system_general} is uniquely solvable as soon as the points $X_N$ are pairwise distinct.

This example raises the following question:
\begin{question}
\label{question:01}
Does this procedure -- fixing a basis, and then doing interpolation with arbitrary points $X_N$ -- always work, also in higher dimensions? 
\end{question}

The answer to this question will be \textit{no}, as we will see in the Mairhuber Curtis theorem \Cref{asdf}.
In order to state and proof this theorem, we first need to introduce some mathematical notation:

\begin{definition}
	Let $\Omega \subset \R^d$ contain at least $N$ points and let $V \subseteq \mathcal{C}(\Omega)$ be an $N$-dimensional linear space. \\
	The space $V$ is called a \underline{Haar space of dimension $N$ on $\Omega$},
	if for arbitrary points $\{ x_i \} \subset \Omega$ and arbitrary values $\{ f_i \}_{i=1}^n \subset \R$ there exists a unique function $s \in V$ that satisfies 
	\begin{align*}
	s(x_i) = f_i \qquad i=1, ..., N.
	\end{align*}
\end{definition}

The polynomial spaces which we encountered before are a first example of such a Haar space:

\begin{example}
Let $\Omega = [a, b] \subset \R$ for $a < b \in \R$. 
Then $V = \pi_{N-1}(\Omega)$ (space of polynomials up to degree $N-1$ on $\Omega$) is a Haar space of dimension $N$ of $\Omega$.
\end{example}

The following proposition gives several equivalent statements on Haar spaces:

\begin{prop}
\label{prop:haar_space}
Let $\Omega \subset \R^d$ contain at least $N$ points and let $V \subseteq \mathcal{C}(\Omega)$ be an $N$-dimensional linear space. \\
Then the following statements are equivalent:
\begin{enumerate}
\item $V$ is an $N$-dimensional Haar space on $\Omega$
\item Every $v \in V \setminus \{ 0 \}$ has at most $N-1$ distinct zeros
\item For any set of pairwise distinct points $X_N \subset \Omega$ and any basis $\{ \phi_j \}_{j=1}^N$ of $V$, 
the interpolation matrix $A_{\phi, X_N}$ is invertible, i.e.\ $\det(A_{\phi, X_N}) \neq 0$.
\end{enumerate}
\end{prop}


\begin{proof}
\begin{itemize}
\item 1. $\Rightarrow$ 2.: We assume that $V$ is an $N$-dimensional Haar space on $\Omega$, 
and that there is a $v \in V \setminus \{ 0 \}$ with (at least) $N$ distinct zeros $X_N \subset \Omega$.
Additionaly we consider $u := 0$ (zero-function) and observe that $u \in V$ (as $V$ is a linear space).
Especially we have $u \neq v$.
Now, both $u$ and $v$ are distinct interpolants for the points $X_N$ with target values $f_i = 0$ for $i=1, ..., N$,
which is a contradiction to the uniqueness of the interpolant within the definition of a Haar space.
\item 2. $\Rightarrow$ 3.: We assume that there is a set of pairwise distinct points $X_N \subset \Omega$ and a basis $\{ \phi_j \}_{j=1}^N$ of $V$ such that the interpolation matrix is not invertible, i.e.\ it is singular.
Then there is a vector $\alpha \in \R^N \setminus \{ 0 \}$ such that $A_{\phi, X_N} \alpha = 0$.
As $\alpha \neq 0$, the function $v(x) := \sum_{j=1}^N \alpha_j \phi_j(x)$ is not the zero function.
However $v$ has $N$ distinct zeros (namely in the points $X_N$), which is thus a contradiction:
\begin{align*}
v(x_i) = \sum_{j=1}^N \alpha_j \phi_j(x_i) = (A_{\phi, X_n} \alpha)_i = 0 \qquad \text{for all } i=1, ..., N.
\end{align*}
\item 3. $\Leftrightarrow$ 1.: The matrix $A_{\phi, X_N}$ is invertible if and only if $A_{\phi, X_N} \alpha = b$ has a unique solution of any $b = \{f_i \}_{i=1}^N$,
which holds if and only if $\sum_{j=1}^N \alpha_j \phi_j(x_i) = f_i, i=1, ..., N$ has a unique solution $\alpha = \{ \alpha_j \}_{j=1}^N$,
which holds if and only if $s(x) := \sum_{j=1}^N \alpha_j \phi_j(x)$ is the unique interpolant.
\end{itemize}
\end{proof}

With these background knowledge on Haar spaces, we can frame \Cref{question:01} more precisely:

\begin{question}
\label{question:02}
Is there any Haar space in dimension $d>1$?
\end{question}
As announced already before, the answer to this question will be no, as established by the famous Mairhuber-Curtis theorem:


\begin{thm}[Mairhuber-Curtis]
\label{th:mairhuber_curtis}
Let $\Omega \subset \R^d$, $d>1$ be a set with nonempty interior.
Then there exists no Haar space of dimension $N>1$ on $\Omega$.
\end{thm}

\begin{proof}
For any $N$-dimensional space $V \subset \mathcal{C}(\Omega)$, we will show the existence of a set $X_N \subset \Omega$ of pairwise distinct points, 
such that the interpolation matrix has determinant zero and is thus singular, 
which is a contradiction to \Cref{prop:haar_space}:

We assume $V:= \mathrm{span} \{ \phi_1, ..., \phi_N \} \subset \mathcal{C}(\Omega)$ to be a Haar space of dimension $N > 1$ on $\Omega$.
As $\Omega$ has a nonempty interior, there exists a ball $B(x_0, \varepsilon) \subset \Omega$ for some $x_0 \in \Omega, \varepsilon > 0$.
Now we consider pairwise distinct points $X_N = \{ x_1, ..., x_N \} \subset \Omega$ with $x_1, x_2 \in B(x_0, \varepsilon)$.
As $V$ is assumed to be a Haar space, we have $\det(A_{\phi, X_N}) \neq 0$ by \Cref{prop:haar_space}. 

Now we consider continuous curves $\gamma_1, \gamma_2: [0, 1] \rightarrow B(x_0, \varepsilon) \subset \Omega$ with $\gamma_1(0) = \gamma_2(1) = x_1$, $\gamma_2(0) = \gamma_1(1) = x_2$ that do not intersect in any other points, 
that do not intersect themselves and that have no points in common with $\{x_3, ..., x_N\}$.
This is possible due to the assumption $d>1$.
Like this, $X_N(t) := \{ \gamma_1(t), \gamma_2(t), x_3, ..., x_N \}$ are pairwise distinct for any $t \in [0, 1]$.
Due to \Cref{prop:haar_space}, we have $\det(A_{\phi, X_N(1)}) \neq 0$.

Now we consider the function $D: [0, 1] \rightarrow \R, t \mapsto \det(A_{\phi, X_N(t)})$ and show that it changes sign:
As $A_{\phi, X_N} = A_{\phi, X_N(0)}$ and $A_{\phi, X_N(1)}$ only differ by a permuation of the first two rows, 
we have $D(1) = -D(0)$ and thus $D(0) D(1) < 0$.
As the function $D(t)$ is continuous, there needs to be a $\tilde{t} \in (0, 1)$ such that $D(\tilde{t}) = 0$.
This means that $X_N(\tilde{t})$ is a set of pairwise distinct points s.t.\ $A_{\phi, X_N(\tilde{t})}$ has determinant zero, which is a contradiction to \Cref{prop:haar_space}.
\end{proof}


The Mairhuber-Curtis theorem from \Cref{th:mairhuber_curtis} essentially tells us, 
that it is not possible to choose the approximation space $V$ a-priori, i.e.\ independent of the points $X_N$. 
In order to solve this issue, we will make use of kernels, which will allow us to use data-dependent approximation spaces.



\section{Kernels}

In this section, we will introduce the notion of a kernel as well as special classes of kernels 
and discuss their basic properties.
Finally we will see how interpolation can be done with kernels in a well defined way, circumventing the Mairhuber-Curtis theorem \Cref{th:mairhuber_curtis}.

We start with the basic definition of a kernel:

\begin{definition}
Let $\Omega$ be a nonempty set. 
\begin{itemize}
\item A \underline{real-valued kernel on $\Omega$} is a symmetric function $k: \Omega \times \Omega \rightarrow \R$.
\item A \underline{complex-valued kernel on $\Omega$} is a hermitian function $k: \Omega \times \Omega \rightarrow \R$.
\end{itemize}
\end{definition}

In the following we will mainly focus on real-valued kernels and on kernel which are defined on $\Omega \subseteq \R^d$.
However we note that kernels can be defined on very general sets that need not posses any structure, such as graphs, molecules or images.

We continue with the definition of a kernel matrix, 
which enables us to introduce the notion of positive definite (PD) and strictly positive-definite (SPD) kernels:

\begin{definition}
\label{def:positive_def_kernels}
Let $\Omega$ be a nonempty set.
\begin{itemize}
\item For any $N \in \N$ and any set of $N$ pairwise distinct points $X_N := \{ x_i \}_{i=1}^N \subset \Omega$, the \underline{kernel matrix} $A := A_{k, X_N} \in \R^{N \times N}$ is defined as $A := [k(x_i, x_j)]_{i,j=1}^N$.
\item A kernel $k$ on $\Omega$ is \underline{positive definite} (PD), if for all $N \in \N$ and all pairwise distinct points $X_N$ the kernel matrix $A$ is positive semi-definite, i.e.\
\begin{align*}
\textstyle \sum_{i,j=1}^N \alpha_i \alpha_j k(x_i, x_j) = \alpha^\top A \alpha \geq 0 \qquad \forall 0 \neq \alpha = \{ \alpha_i \}_{i=1}^N \in \R^N.
\end{align*}
\item A kernel $k$ on $\Omega$ is \underline{strictly positive definite} (SPD), if for all $N \in \N$ and all pairwise distinct points $X_N$ the kernel matrix $A$ is positive definite, i.e.\
\begin{align*}
\textstyle \sum_{i,j=1}^N \alpha_i \alpha_j k(x_i, x_j) = \alpha^\top A \alpha > 0 \qquad \forall 0 \neq \alpha = \{ \alpha_i \}_{i=1}^N \in \R^N.
\end{align*} 
\end{itemize}
\end{definition}

There is another important class of kernels, namely conditionally positive definite kernels, which are however not discussed in the following.
It is worth to note, that some literature (e.g.\ \cite{wendland2005scattered}) does not use ``\textit{strictly} positive definite'' kernels, but uses directly ``positive definite'' kernels.

In order to show why the notion of a strictly positive definite kernel is important, 
we recall the following statement from linear algebra:

\begin{prop}
\label{prop:linear_algebra}
Let $A \in \R^{N \times N}$ be a symmetric matrix. 
Then the following are equivalent:
\begin{enumerate}
\item $A$ is positive definite, i.e.\ $\alpha^\top A \alpha > 0$ for all $0 \neq \alpha \in \R^N$.
\item The eigenvalues $\{ \lambda_i \}_{i=1}^N$ of $A$ are positive
\end{enumerate}
In this case, it holds:
\begin{itemize}
\item $A$ is invertible with $\det(A) > 0$.
\end{itemize}
\end{prop}

\begin{proof}
As $A$ is symmetric, the spectral theorem gives an eigen-decomposition as $A = V\Lambda V^\top$ with $V, \Lambda \in \R^{N \times N}$, $V$ orthogonal and $\Lambda$ diagonal with the eigenvalues on its diagonal:
\begin{itemize}
\item 1. $\Rightarrow$ 2.: As $V$ is orthogonal, its columns $V_i$ form an orthonormal basis. Thus we can decompose any $0 \neq \alpha \in \R^N$ as $\alpha = \sum_{i=1}^N \beta_i V_i = V\beta$. Then we calculate:
\begin{align*}
\alpha^\top A \alpha = (\beta^\top V^\top)A(V \beta) = \beta^\top V^\top V \Lambda V^\top V \beta = \beta^\top \Lambda \beta = \sum_{i=1}^N \beta_i^2 \lambda_i.
\end{align*}
Thus $\lambda_i > 0$ implies $\alpha^\top A \alpha > 0$ for all $\alpha \neq 0$.
\item 2. $\Rightarrow$ 1.: We have $\alpha^\top A \alpha > 0$ for any $\alpha \neq 0$, thus we pick $\alpha := V_i$ (column of $V$ as specified above), and have
\begin{align*}
0 < V_i^\top A V_i \lambda_i V_i^\top V_i = \lambda_i.
\end{align*}
\end{itemize}
If all the eigenvalues are positive, also the determinant is positive due to $\det(A) = \prod_{i=1}^N \lambda_i$.
\end{proof}

\Cref{def:positive_def_kernels} in conjunction with \Cref{prop:linear_algebra} now allows us to show how strictly positive definite kernels can be used for our inteprolation problem \Cref{prob:interpolation_problem}:


\begin{thm}
Let $\Omega \subset \R^d$ and let $k$ be a SPD kernel on $\Omega$. 
For any set $\{ x_i \}_{i=1}^N \subset \Omega$ of pairwise distinct points and corresponding target values $\{ f_i \}_{i=1}^N \subset \R$,
there exists a unique kernel interpolant
\begin{align*}
s(x) := \sum_{j=1}^N \alpha_j k(x, x_j)
\end{align*}
that satisfies $s(x_i) = f_i$ for all $i=1, ..., N$. 
\end{thm}

\begin{proof}
The interpolation conditions $s(x_i) = f_i$ for all $i=1, ..., N$ yield to the linear system of equations
\begin{align}
\label{eq:lin_eq_system_kernel}
A_{k, X_N} \alpha := 
\begin{pmatrix}
k(x_1, x_1) & k(x_1, x_2) & \dots & k(x_1, x_N) \\
k(x_2, x_1) & k(x_2, x_2) & \dots & k(x_2, x_N) \\
\vdots & \vdots & \vdots & \vdots \\
k(x_N, x_1) & k(x_N, x_2) & \dots & k(x_N, x_N) \\
\end{pmatrix}
\begin{pmatrix}
\alpha_1 \\ \alpha_2 \\ \vdots \\ \alpha_N
\end{pmatrix}
=
\begin{pmatrix}
f_1 \\ f_2 \\ \vdots \\ f_N
\end{pmatrix}.
\end{align}
The matrix $A_{k, X_N}$ is exactly the kernel matrix from \Cref{def:positive_def_kernels},
and due to the assumption on the kernel to be SPD, this matrix is positive definite.
By \Cref{prop:linear_algebra} we have $\det(A_{k, X_N}) > 0$, thus this linear system of equations is uniquely solvable.
\end{proof}

As we saw that (strictly positive definite) kernels can be used for interpolation, 
we will give two examples of kernels:


\begin{example}
\label{ex:kernels}
\begin{itemize}
\item The linear kernel is defined as $k: \R^d \times \R^d: (x, z) \mapsto (x, z)_{\R^d}$. 
It simply returns the inner product of two points $x, z \in \R^d$.
It is easy to see, that the linear kernel is positive definite kernel, but not strictly positive definite.
\item The Gaussian kernel is defined as $k: \R^d \times \R^d: (x, z) \mapsto \exp(-\Vert x - z \Vert^2)$.
It makes use of the negative squared exponential of the distance between two data points, and is thus a first example of a ``radial basis function kernel''.
The Gaussian kernel can be shown to be strictly positive definite.
\end{itemize}
\end{example}



The linear kernel was a first example of a kernel, which is defined with help of a ``feature map''.
For these kind of kernels, one can directly show the positive definiteness:

\begin{thm}
Let $\Omega$ be a set, $H$ a Hilbert space and $\phi: \Omega \rightarrow H$ a mapping (\underline{feature map}).
Then the kernel 
\begin{align*}
k(x, z) := (\phi(x), \phi(z))_{H}
\end{align*}
is positive definite.
\end{thm}

\begin{proof}
We verify the condition from the definition in \Cref{def:positive_def_kernels}:
\begin{align*}
\sum_{i,j=1}^N \alpha_i \alpha_j k(x_i, x_j) &= \sum_{i, j=1}^N \alpha_i \alpha_j (\phi(x_i), \phi(x_j))_H 
= (\sum_{i=1}^N \alpha_i \phi(x_i), \sum_{j=1}^N \alpha_j \phi(x_j) )_H \\
&= \left \Vert \sum_{i=1}^N \alpha_i \phi(x_i) \right \Vert_H^2 \geq 0.
\end{align*}
\end{proof}



We continue with some further elementary properties on kernels:

\begin{prop}
Let $k: \Omega \times \Omega \rightarrow \R$ be a PD kernel. Then:
\begin{itemize}
\item Non-negativity of the diagonal: 
\begin{align*}
k(x, x) \geq 0 \qquad \text{for all ~} x \in \Omega
\end{align*}
\item Cauchy-Schwarz inequality: 
\begin{align*}
k(x, z)^2 \leq k(x, x)k(z,z) \qquad \text{for all ~} x, z \in \Omega.
\end{align*}
\end{itemize}
If $k$ is SPD, then the same holds using ``$>$'' instead of ``$\geq$''.
\end{prop}

\begin{proof}
\begin{itemize}
\item We consider $N := 1$ and $X_1 = \{ x\}, \alpha = 1$. 
Then the positive definiteness implies $\alpha^\top A \alpha = k(x, x) \geq 0$.
\item We consider $N := 2$ and $X_2 = \{ x, z \}$. 
Then we have
\begin{align*}
A_{k, X_N} = \begin{pmatrix}
k(x, x) & k(x, z) \\
k(z, x) & k(z, z).
\end{pmatrix}
\end{align*}
By \Cref{prop:linear_algebra} we have $0 \leq \det(A_{k, X_N}) = k(x, x) k(z, z) - k(x, z)^2$, which can be rearranged to give the statement.
\end{itemize}
\end{proof}

The following proposition allows us to build new kernels, while maintaining their (strict) positive definiteness properties:

\begin{prop}
\label{prop:build_kernels}
Let $k_1, k_2: \Omega \times \Omega \rightarrow \R$ be PD kernels,
$f: \Omega \rightarrow \Omega$ and $g: \Omega \rightarrow \R$, $\Omega' \subset \Omega$. 
Then the following kernels are also PD (SPD):
\begin{enumerate}
\item $k: \Omega' \times \Omega' \rightarrow \R$ with $k := (k_1)|_{\Omega' \times \Omega'}$ \hfill (SPD if $k_1$ is SPD)
\item $k(x, z) := k_1(x, z) + k_2(x, z)$ \hfill (SPD if $k_1$ or $k_2$ is SPD)
\item $k(x, z) := \alpha k_1(x, z)$ for $\alpha \geq 0$ \hfill (SPD if $k_1$ is SPD and $\alpha > 0$)
\item $k(x, z) := k_1(x, z) k_2(x, z)$ \hfill (SPD if both $k_1$ and $k_2$ are SPD)
\item $k(x, z) := \exp(k_1(x, z))$ 
\item $k(x, z) := k_1(f(x), f(z)))$ \hfill (SPD if $k_1$ is SPD and $f$ is injective)
\item $k(x, z) := g(x)g(z)$
\item $k(x, z) := g(x)k_1(x, z)g(z)$ \hfill (SPD if $k_1$ is SPD and $g(x) \neq 0$)
\end{enumerate}

\end{prop}

The proofs are rather elementary and mostly directly follow from the definitions -- and are thus left as a (voluntary) homework.

Finally we will see, that the Gaussian kernel from \Cref{ex:kernels} is SPD:

\begin{prop}
The Gaussian kernel $k: \Omega \times \Omega \rightarrow \R$ for $\Omega \subset \R^d$ given as
\begin{align*}
k(x, z) = \exp(-\Vert x - z \Vert^2)
\end{align*}
is strictly positive definite.
\end{prop}

\begin{proof}
We start with the case $d=1$.
We will make use of the Fourier transform and its inverse, which are given as
\begin{align*}
\mathcal{F}[f](\omega) &:= \frac{1}{\sqrt{2\pi}} \cdot \int_{-\infty}^\infty \exp(-ix\omega) f(x) ~ \mathrm{d}x \\
\mathcal{F}^{-1}[\hat{f}](x) &:= \frac{1}{\sqrt{2\pi}} \cdot \int_{-\infty}^\infty \exp(ix\omega) \hat{f}(\omega) ~ \mathrm{d}\omega
\end{align*}
For our Gaussian kernel, we recall the following:
\begin{align*}
\mathcal{F} \left[\exp(-x^2) \right](x) &=  \frac{1}{\sqrt{2}} \cdot \exp(-\omega^2 / 4) =: \hat{\Phi}(\omega) \quad \Leftrightarrow \quad
\exp(-x^2) = \mathcal{F}^{-1} \left[\frac{1}{\sqrt{2}} \cdot \exp(-\omega^2 / 4) \right](x).
\end{align*}
With this, we can straightforward compute:
\begin{align*}
\alpha^\top A_{k, X_N} \alpha &= \sum_{j, l=1}^N \alpha_j \alpha_l k(x_j, x_l) = \sum_{j, l=1}^N \alpha_j \alpha_l e^{-(x_j - x_l)^2} \\
&= \frac{1}{\sqrt{2\pi}} \cdot \sum_{j, l=1}^N \alpha_j \alpha_l \int_{-\infty}^\infty e^{i(x_j - x_l)\omega} \hat{\Phi}(\omega) ~ \mathrm{d}\omega \\
&= \frac{1}{\sqrt{2\pi}} \cdot \int_{-\infty}^\infty \hat{\Phi}(\omega) \cdot \left| \sum_{j=1}^N \alpha_j e^{ix_j \omega} \right|^2 ~ \mathrm{d}\omega.
\end{align*}
Since $\hat{\Phi}(\omega) > 0$ and $\left| \sum_{j=1}^N \alpha_j e^{ix_j \omega} \right|^2 > 0$ a.e.\ (as $X_N$ are pairwise distinct),
we have $\alpha^\top A_{k, X_N} \alpha$.

The general case $d>0$ follows from the observation
\begin{align*}
\exp(-\Vert x - z \Vert^2) = \exp(-\sum_{i=1}^d (x^{(i)} - z^{(i)})^2) = \prod_{i=1}^d \exp(-(x^{(i)} - z^{(i)})^d)
\end{align*}
and an application of \Cref{prop:build_kernels}.

\end{proof}

It the previous proof it is worth to note, that the whole proof works as long as $\hat{\Phi}(\omega) > 0$, i.e.\ it can also be leveraged for other kernels.
This is indeed done for so-called ``translational invariant'' or ``radial basis funtion'' kernels.







\section{Reproducing kernel Hilbert spaces}

In the previous section we saw that kernel interpolation is well defined.
Now we want to consider the approximation of functions with kernel interpolation.
The properties of this approximation will for sure depend on the class of functions and on the kernel which we consider. 
In the following we will see, that there is a ``native space'' of functions, for which this works well, the so-called reproducing kernel Hilbert space (RKHS).
Within this space, we will be able to derive a  comprehensive analysis.
We start with the definition of a reproducing kernel Hilbert space:


\begin{definition}
\label{def:rkhs}
Let $\Omega$ be a nonempty set, $\mathcal{H}$ a Hilbert space of function $f: \Omega \rightarrow \R$ with inner product $(\cdot, \cdot)_{\Ha}$. \\
$\Ha$ is called a \underline{Reproducing Kernel Hilbert space} (RKHS) on $\Omega$,
if there exists a function $k: \Omega \times \Omega \rightarrow \R$ (the \underline{reproducing kernel}) such that
\begin{enumerate}
\item $k(\cdot, x) \in \Ha$ for all $x \in \Omega$
\item $(f, k(\cdot, x))_{\Ha} = f(x)$ for all $x \in \Omega, f \in \Ha$ (\underline{reproducing property})
\end{enumerate}
\end{definition}

The first point ensures, that the kernel (with fixed second argument) is a function itself within this Hilbert space of functions.
The second point, the so called reproducing property, shows that the Riesz representer of the point evaluation functional $\delta_x$ is given by $k(\cdot, x)$.

We continue with an easy example of an RKHS on $\Omega = (0, 1)$:

\begin{example}
Consider $\Omega = (0, 1)$ and the the Sobolev space $H_0^1(\Omega)$:
\begin{align*}
H_0^1(\Omega) := \{ f: [0, 1] \rightarrow \R, f \in L^2(\Omega), f' \in L^2(\Omega), f(0) = f(1) = 0 \}
\end{align*}
with inner product defined as
\begin{align*}
(f, g)_{H_0^1(\Omega)} := \int_0^1 f'(y) g'(y) ~ \mathrm{d}y
\end{align*}
Then $H_0^1(\Omega)$ is a RKHS on $\Omega$ with reproducing kernel given by
\begin{align*}
k(x, z) = \min(x, z) - xz = 
\begin{cases}
x(1-z) & \quad x < z \\
z(1-x) & \quad x \geq z.
\end{cases}
\end{align*}

Indeed we can quickly verify the two properties of an RKHS from \Cref{def:rkhs}: 
For the first property, we can straightforward compute $\frac{\mathrm{d}}{\mathrm{d}y} k(y, x)$ and $k(0, x)$ and $k(1, x)$ to see that $k(\cdot, x) \in H_0^1(\Omega)$.
For the second property, we calculate for any $f \in H_0^1(\Omega)$:
\begin{align*}
(f, k(\cdot, x))_{H_0^1(\Omega)} &= \int_0^1 f'(z) \frac{\mathrm{d}}{\mathrm{d}z} k(z, x) ~ \mathrm{d}z \\
&= \int_0^x f'(z) \frac{\mathrm{d}}{\mathrm{d}z} k(z, x) ~ \mathrm{d}z + \int_x^1 f'(z) \frac{\mathrm{d}}{\mathrm{d}z} k(z, x) ~ \mathrm{d}z \\
&= \int_0^x f'(z) (1-x) ~ \mathrm{d}z + \int_x^1 f'(z) (-x) ~ \mathrm{d}z \\
&= (1-x) (f(x) - f(0)) - x (f(1) - f(x)) \\
&= f(x),
\end{align*}
where we used $f(0) = f(1) = 0$ in the final step.
\end{example}

The following proposition shows that linear combinations of $k(\cdot, x_i)$ are included in the RKHS, and how to compute the inner products of such linear combinations. 
Both properties directly follow from the definition of the RKHS by leveraging the linearity of Hilbert spaces.

\begin{prop}
Let $\Ha$ be a RKHS on $\Omega$ with reproducing kernel $k$. 
Let $N, M \in \N$, $\alpha \in \R^N, \beta \in \R^M$ and $X_N, Z_M \subset \Omega$ and consider
\begin{align*}
f = \sum_{i=1}^N \alpha_i k(\cdot, x_i), ~ g = \sum_{j=1}^M \beta_j k(\cdot, z_j).
\end{align*}
Then it holds
\begin{enumerate}
\item $f, g \in \Ha$,
\item $(f, g)_{\Ha} = \sum_{i=1}^N \sum_{j=1}^M \alpha_i \beta_j k(x_i, z_j)$ 
\end{enumerate}
\end{prop}

\begin{proof}
\begin{enumerate}
\item By the first property within the definiton of the RKHS in \Cref{def:rkhs}, 
we have that $k(\cdot, x_i) \in \Ha$ for all $i=1, ..., N$ as well as $k(\cdot, z_j) \in \Ha$ for all $j = 1, ..., M$.
Since $H$ is a Hilbert space (by definition), it is a linear space, 
and thus also the linear combinations $f$ and $g$ are within $\Ha$.
\item To compute the inner product between $f$ and $g$, we leverage the reproducing property (second property within \Cref{def:rkhs} and make use of the linearity of the inner product:
\begin{align*}
(f, g) &= (\sum_{i=1}^N \alpha_i k(\cdot, x_i), \sum_{j=1}^M \beta_j k(\cdot, z_j))_{\Ha} \\
&= \sum_{i=1}^N \sum_{j=1}^M \alpha_i \beta_j (k(\cdot, x_i),  k(\cdot, z_j))_{\Ha} \\
&= \sum_{i=1}^N \sum_{j=1}^M \alpha_i \beta_j k(x_i, z_j).
\end{align*}
\end{enumerate}

\end{proof}




%
%
%
%% \section{Basic properties and examples}
%\section{Kernels}
%
%This lecture: Definition of kernels, kernel interpolation, basic properties
%
%\begin{itemize}
%\item Definition kernel, kernel matrix, pd and spd
%\item (criteria for spd)
%\item Theorem: Kernel interpolation is well defined
%\item properties of (s)pd kernels
%\item basic operations of kernels (sum, scalar, product, exponential, ...
%\item examples of kernels: linear kernel, Gaussian kernel + Fourier + remark for RBF kernels, kernels via feature maps
%\end{itemize}


%Section 2: Basic properties and examples of kernels
%	- 2.1 Criteria for (s)pd kernels
%			Prop 2.3: i) and iii)
%	- 2.2 Properties of (s)pd kernels
%	- 2.3 Basic operations on kernels
%			o Gaussian kernel


% \section{Kernels and Hilbert spaces}
%\section{Reproducing kernel Hilbert spaces}
%
%\begin{itemize}
%\item definition RKHS + example $H_0^1$
%\item Proposition 3.5 + Characterizations
%\item Theorem Aronszajn: Existence of RKHS
%\item Feature map: Kernel feature map, other feature maps (Gaußkern, $L^2$ feature space)
%\item RKHS and smoothness, basic operations on RKHS
%\end{itemize}
%
%
%%Section 3: Kernels and Hilbert spaces
%%	- 3.1: Reproducing kernel Hilbert spaces
%%			o Definition 3.2 
%%			o Theorem 3.6
%%	- 3.2: The native space of a PD kernel
%%			o Theorem 3.10 (Aronszajn)
%%			o Prop 3.12: Kernel feature map
%%			o Prop 3.14 i)
%%			o Prop 3.15: Basic operations on kernels
%
%\section{Interpolation in RKHS}
%
%\begin{itemize}
%\item Interpolation as orthogonal projection + orthogonal decomposition
%\item Interpolation gives best approximation + minimal norm interpolation
%\item Error bounds: Definition power function, alternative form, basic properties
%\end{itemize}
%
%
%
%%Section 4: Interpolation in native spaces
%%			o Prop 4.1 (Interpolation in the native space)
%%			o Cor 4.2 (Orthogonal decomp)
%%	- 3.1: Optimality of kernel interpolation
%%			o Prop 4.3 (Interpolation gives best approximation)
%%			o Prop 4.4 (Minimal norm interpolation)
%
%
%\section{Kernels and Neural networks}
%
%\begin{itemize}
%\item Remark that we aim at providing some intuition, without discussing all mathematical details.
%It is about ongoing research
%\item (Short discussion of NNGP)
%\item Jupyter notebooks: Empirical findings: Weights change only little
%% \item Follow https://pages.cs.wisc.edu/~yliang/cs839_spring22/documents/0301-lec11-NTK-1.pdf
%\item Neural network training, solving the ODE (follow Section ``Gradient Flow'' within \url{https://rajatvd.github.io/NTK/}) + remark/discussion in which limits this hold
%\item NTK for two-layer NN
%\item Jupyter notebook: Show that training and NTK prediction actually match
%\end{itemize}
%


%----------------------------------------------------------------------------------------
