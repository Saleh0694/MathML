% Chapter 1
% !TeX spellcheck = en_US 
\chapter{Introduction} % Main chapter title

\label{Chapter1} % For referencing the chapter elsewhere, use \ref{Chapter1} 
\setcounter{chapter}{1}
%----------------------------------------------------------------------------------------

% Define some commands to keep the formatting separated from the content 
\newcommand{\keyword}[1]{\textbf{#1}}
\newcommand{\tabhead}[1]{\textbf{#1}}
\newcommand{\code}[1]{\texttt{#1}}
\newcommand{\file}[1]{\texttt{\bfseries#1}}
\newcommand{\option}[1]{\texttt{\itshape#1}}
Underlying the success of artificial intelligence are learning algorithms, i.e.,
algorithms that learn from data to perform a certain task. We start this book by
two concrete examples of supervised learning algorithms. In the first example we
consider the problem of approximating functions from pointwise evaluations using linear regression. In the second example we look at the task of
classifying hand-written digits. In these two examples we identify and
familiarize ourselves with the main components of learning algorithms;
\emph{datasets}, \emph{a hypothesis class}, and \emph{optimization algorithms}.
We further identify important aspects of supervised learning algorithms, such as
overfitting, and underfitting. Finally, we motivate in these two examples
problems at the forefront of research in mathematical machine learning, namely
\emph{the curse of dimensionality (CoD)} and double/multiple descent phenomenon. 
 
\section{Examples of supervised learning}
\cite{Saleh:JCP155:144109}
Approximating smooth functions from point evaluations

Classifying hand-written digits. 

\subsection{Curse of Dimensionality}
\subsection{Underfitting, overfitting, or just right?}
\subsection{Approximating highly-oscillatory functions}

\section{A formal definition of learning}

\section*{Wait! What is what?}
Here is a list of questions that help you check your understanding of key
concepts inside this chapter?

%----------------------------------------------------------------------------------------
